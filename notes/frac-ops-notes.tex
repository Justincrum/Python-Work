\documentclass[10pt,a4paper]{article}
\usepackage{amsmath,amssymb,amsfonts,amsthm}
%\usepackage{mathabx,MnSymbol}
\usepackage{float,color}
%\usepackage{graphicx}
\usepackage[all]{xy}
\usepackage{enumitem}

% Graphics packages
\usepackage{color} % black,white,red,green,blue,cyan,magenta,yellow
\usepackage{ifpdf}
\ifpdf
    % PDF Graphics packages
    \usepackage[pdftex]{graphicx}
    %\usepackage{pdftricks}
    %\begin{psinputs}
    %  \usepackage{pstricks}
    %\end{psinputs}
    \usepackage[pdftex]{hyperref}
    %\usepackage{hyperref}
    \hypersetup{
        %bookmarks=true,        % show bookmarks bar?
        unicode=false,          % non-Latin characters in Acrobat� bookmarks
        pdftoolbar=true,        % show Acrobat� toolbar?
        pdfmenubar=true,        % show Acrobat menu?
        pdffitwindow=false,     % window fit to page when opened
        pdfstartview={FitH},    % fits the width of the page to the window
%        pdftitle={AKG Article},      % title
%        pdfauthor={Andrew Gillette},   % author
%        pdfsubject={Mathematics},    % subject of the document
%        pdfcreator={Andrew Gillette},  % creator of the document
%        pdfproducer={Andrew Gillette}, % producer of the document
        pdfkeywords={math, mathematics}, % list of keywords
        pdfnewwindow=true,      % links in new window
        colorlinks=true,        % false: boxed links; true: colored links
        linkcolor=red,          % color of internal links
        citecolor=blue,         % color of links to bibliography
        filecolor=magenta,      % color of file links
        urlcolor=cyan           % color of external links
    }
    \def\myfigpng#1#2{\includegraphics[height=#2]{#1.png}}
    \def\myfigpdf#1#2{\includegraphics[height=#2]{#1.pdf}}
    \def\myfigps#1#2{\includegraphics[height=#2]{#1.pdf}}
    \def\myfig#1#2{\includegraphics[height=#2]{#1.png}}
    \typeout{====== Invoked by pdflatex ======================}
\else
    % DVI Graphics packages
    \usepackage{graphicx}
    \usepackage{pstricks}
    \newenvironment{pdfpic}{}{}
    \newcommand{\href}[2]{#2}
    \def\myfigpng#1#2{\includegraphics[height=#2]{#1}} % auto-select .ps/.eps
    \def\myfigpdf#1#2{\includegraphics[height=#2]{#1}} % auto-select .ps/.eps
    \def\myfigps#1#2{\includegraphics[height=#2]{#1}}  % auto-select .ps/.eps
    \def\myfig#1#2{\includegraphics[height=#2]{#1}}    % auto-select .ps/.eps
    \typeout{====== Invoked by latex ======================}
\fi


%\newcommand{\red}[1]{{\textcolor{red}{\textbf{#1}}}}

%---------------------------------------------------


\newcommand{\red}[1]{\textcolor{red}{#1}}
\newcommand{\ldeg}{\textnormal{ldeg}}


%-----------------------------------------

%\setlength{\textwidth}{1.1\textwidth}

\numberwithin{equation}{section}

\begin{document}
\title{Fractional Derivatives for Geometry Processing}


\author{Justin Crum, Andrew Gillette and Josh Levine
\thanks{University of Arizona, Tucson, Arizona, USA}
}

%\date{}

\maketitle

Some ideas to pursue going forward
\begin{itemize}
\item Suppose we have a coarse and fine-grained mesh, e.g. a decimated version of a fine mesh.  Are eigenvalues / vectors preserved at all?  Do fractional operators help pick out key features in the coarse mesh?
\item If we ``crop'' the eigenvalue spectrum, do we still get useful information from the eigenvalues we keep?
\item Can fractional eigenvalues/vectors be used for ridge detection in a 2D mesh embedded in 3D?
\item Compare 1 $h$-step of the regular Laplacian smoothing to 2 $h^{\frac 12}$ steps of a $\frac 12$ Laplacian smoothing. 
We see that:\\[2mm]
\begin{tabular}{ll}
1 $h$ step $\Delta$ & $I+ h\Delta$ \\
2 $h^{\frac 12}$ steps $\Delta^{\frac 12}$ & $I + h\Delta + 2h^{\frac 12}\Delta^{\frac 12}$
\end{tabular}\\[2mm]
The difference is $2h^{\frac 12}\Delta^{\frac 12}$.  What does this encode about the geometry?
\item Looking at the paper ``Spectral geometry processing with manifold harmonics'' by Vallet and L\'evy (2008) in Google Scholar and searching its ``cited by'' list for the term ``fractional'' turns up few results.  One paper, by Hu, Wang and Qin claims to investigate ``the relationship between Riesz transform and fractional Laplacian operator, which can enable the computation of Riesz transform on surfaces via eigenvalue decomposition of Laplacian matrix.''
\item This survey article may be a useful read:\\
What Is the Fractional Laplacian?\\
\url{https://arxiv.org/abs/1801.09767}


\end{itemize}

\end{document}
